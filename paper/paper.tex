\documentclass[a4paper,12pt,parskip]{article}

\nonstopmode

\usepackage{graphicx}
\usepackage{subfig}
\usepackage{polyglossia}

\usepackage{amsfonts}
\usepackage{amsmath}
\usepackage{amssymb}

\usepackage[hidelinks]{hyperref}

\usepackage{fontspec}

\usepackage{algorithm,algorithmicx}
\usepackage[noend]{algpseudocode}

\usepackage{fancyhdr}
\pagestyle{fancy}
\cfoot{\thepage}
\lhead{InformatiCup2022}
\chead{Team \textit{Off-by-one}}
\rhead{\nouppercase{\leftmark}}

\linespread{1.06}
\setlength{\parskip}{6pt plus2pt minus2pt}
\setcounter{tocdepth}{3}

\widowpenalty 10000
\clubpenalty 10000

\title{An Application Of Tabu-Enhanced Genetic Search To A Railway Optimization Problem}
\author{
    \\ [1cm]
    \Large Team Off-by-one \\ [0.5cm]
    \normalsize Consisting of \\ [0.5cm]
    \normalsize Lennart Carstens-Behrens \\
    \texttt{lennart.carstens-behrens@stud.htwk-leipzig.de} \\
    \normalsize Oliver Herrmann \\
    \texttt{oliver.herrmann@stud.htwk-leipzig.de} \\ [0.5cm]
    \normalsize Students of the \\ [0.5cm]
    \normasize HTWK Leipzig \\ [1.5cm]
}
\date{\today}

\begin{document}

% Title
\maketitle

\vfill

\begin{abstract}
The informatiCup 2022 introduced a railway optimization problem in which trains 
should bring passengers from one station to another via a railway network with 
as little waiting time as possible. This paper describes an application of 
tabu-enhanced genetic search to the railway optimization problem.
\end{abstract}

\newpage

\tableofcontents

\newpage

\section{Introduction}

The 17'ths informatiCup\footnote{informatiCup \url{https://gi.de/informaticup}} 
by the German Informatics Society (GI\footnote{Gesellschaft Für Informatik \url{ https://gi.de/ }}) - 
started in october 2021 - introduced a railway optimization problem. The 
background to this is, that public rail transport in germany will gradually be 
changed to a coordinated timetable, which aims to reduce passenger waiting times 
by having local and long-distance trains arrive at key hubs at coordinated times. 
This timetable is described by Deutsche Bahn as 
"Taktfahrplan \footnote{Taktfahrplan\url{https://www.deutschlandtakt.de/blog/so-funktioniert-der-taktfahrplan/}}" 
or "coordinated timetable".

\subsection{Task}

The task of the informatiCup 2022 includes the creation of a timetable for a rail 
network with the aim of minimizing the total delay of all passengers. The model 
and the timetable are provided and to be returned in a given format. The formats 
and more details about the task can be found in the GitHub 
repository\footnote{Repository \url{https://github.com/informatiCup/informatiCup2022}}.

\section{Literature Review}

The railway optimization problem falls under the computational time complexity 
class NP-Hard. Which means that no efficient algorithm exists for the problem. 
Knowing that no efficient solution exists is helpful, according to 
Steven S. Skienna \cite{skienna08}, as the algorithm designer can productively 
focus on finding an approximation instead of searching for an algorithm that 
solves the problem effiently in nondeterministic polynominal-time.

\subsection{Approximation Algorithms}

The goal of an approximation algorithm is to quickly find a good but not optimal 
solution which can be improved with longer runtime eventually. Algorithms that 
prooved to work well for optimization problems are Genetic search, Simmulated 
Annealing and Tabu Search.

\subsubsection{Genetic Algorithms}

Genetic algorithms where first suggested as a search method for optimization 
problems in 1975 by J. Holland \cite{holland1975}.

A Genetic algorithm borrows biologic concepts by emulating "survival of the 
fittest" principle. The chance of a solution to "survive" increases with its 
fitness which is given by an objective function. Furthermore surviving solutions
are "mutated" by applying random alterations.

\subsubsection{Simulated Annealing}

Simulated Annealing is inspired by a physical process of cooling in thermodynamics, 
where the energy state of a system is described by the energy state of each 
particle within the system. The energy state of a particle jumps randomly, 
this transition is determined by the temperature of the system. The probability 
that the energy state of a particle changes from $e_i$ and $e_j$ increases with 
decreasing difference $\Delta e = e_i - e_j$ and decreases with decreasing 
temperature $T$. The formula for this transition propability $P(e_i, e_j, T)$ is 
given by

$$
P(e_i, e_j, T) = e^{(e_i, e_j)/k_bT}
$$

where $k_b$ is the so called Boltzmanns constant which defines the dependence of 
the temperature drop on the transition probability.

\subsubsection{Tabu Search}
\label{sec:tabu}

Fred Glover and Eric Taillard specify in \cite{glover1993} that Tabu Search can 
be viewed as an iterative technique which explores a set of problem solutions,
denoted $X$ by repeatedly making moves from one solution $s$ to another solution 
$s'$ located in the neighbourhood $N(s)$ of $s$. Moves are selected according to 
a certain criteria in order to find an optimal solution more quickly. Starting from 
the best solution $s_{best}$, the process is repeated, whereby previously executed 
moves are disalowed (tabu). This prevents cyclic behaviour and not getting stuck 
in suboptimal regions to which the creteria may lead.

\subsection{Which One To Choose}

Steven S. Skienna states in \cite{skienna08}, that he has never encountered any 
problem where genetic algorithms seemed to be the right way to attack these 
problems and he recommends to stick to Simulated Annealing or Tabu Search instead. 
The study \cite{FrancisGorman1998}, on the other hand, shows that genetic 
algorithms enhanced with tabu search can be used effectively specifically for 
railway optimization problems. The tabu restrictions prevents cyclic behaviour by 
searching different directions when a state is revisited. Thus, a tabu-enhanced 
genetic search algorithm, adapted to the task, can be suitable for solving the 
optimization problem.

\section{Models}

This section describes the models and its designations used in the further course 
of this paper. The input model, denoted $\Theta$, holds initial information about 
all entities. The output model, is referred to as timetable, denoted $\Phi$. $\Phi$ 
is given by a set of states $\varphi$ for each point in time. A state $\varphi$ 
holds information about the capacities of each station $C^s$, line $C^l$ and train 
$C^r$ and the location of all trains $L^r$ and passengers $L^p$. 

A list of the indices used is given below. These may change depending on need or 
collision.

\hfill 

\noindent \textbf{Indices}

\noindent \\
$t$ : index for a point in time \\
$s$ : index for a station \\
$l$ : index for a line \\
$r$ : index for a train \\
$p$ : index for a passenger \\

\subsection{Timetable Evaluation}

A timetable $\Phi$ is evaluated by a fitness function $F(\Phi)$ which sums the total 
delay for all passengers. This function is given by

$$
F(\Phi) = \sum_{p=0}^{\rho} max(0, d_{p}) + \sum_{p=0}^{\varrho} t_{max}
$$

where $\rho$ is the number of arrived passengers, $\varrho$ is the number of 
travelling passengers, $d_{p}$ is the delay of the passenger $p$ and $t_{max}$ 
is the latest possible arrival time.

\subsection{Moves}

Furthermore, moves $m \in M^{\varphi}_{r}$ where $M^{\varphi}_r$ is a set of 
all legal moves, can be executed for the train $r$ at state $\varphi$. The 
following moves are possible:

\begin{itemize}
    \item $m^{board} \equiv (t,p,s)$ Boarding passenger $p$ to the train $t$ from station $s$
    \item $m^{detrain} \equiv (t,p,s)$ Detrain passenger $p$ from the train $t$ to station $s$
    \item $m^{depart} \equiv (t,f,d,l)$ Detrain train $t$ from station $f$ to station $d$ via line $l$
    \item $m^{start} \equiv (t,s)$ Start train $t$ on station $s$
    \item $m^{0} \equiv ()$ Null move - nothing happens
\end{itemize}

\section{Algorithm}

The algorithms that are used to find an optimal feasable solution are described 
below. The algorithms form a tabu-enhanced genetic search which has been adapted 
to the specific needs. 

Starting from time $t=0$, a neighbourhood $N(\varphi)$ is scanned for the state 
$\varphi_t$. This is done by evaluating all legal moves $M_r$ for each train $r$. 
The best move $m_{best}$ is picked for each train and pushed to $\varphi$. When 
the best move for each train has been found, $t$ is increased by $1$ and the 
neighborhood is scanned for the next state. This procedure is repeated until any 
of the following events occur:

\begin{itemize}
    \item $t \ge t_{max}$
    \item $\varphi_t$ is illegal
    \item all passengers have arrived
\end{itemize}

Then the genetic search concept of selection is applied by remembering $\Phi$ 
when it is better than all previously visited timetables. Starting from a 
randomly selected state $\varphi \in \Phi_{best}$ the neighourhood is scanned 
again. Executed moves are remembered in a Tabulist $T$ and are excluded from all 
neighbourhoods in the following iterations.

The pseudocodes for search \ref{alg:search} and neighbour \ref{alg:neighbour}
summarize the procedure.

\begin{algorithm}
\caption{search}\label{alg:search}
\begin{algorithmic}
    \State $\textit{Create initial Solution } \Phi$
    \State $\textit{Set best solution } \Phi_{best} = \Phi$
    \State $\textit{Set initial state } \varphi$
    \State $\textit{Set start time } t=0$
    \While {$F(\Phi_{best}) \neq 0$}
        \While {$t \leq t_{max}$}
            \State $\textit{Find the best neighbour } \varphi = neighbour(\varphi)$
            \State $\textit{Push neighbour to solution } \Phi.push(\varphi)$
            \State $\textit{Set the next null state } \varphi = \varphi.next\_null()$

            \If {$\varphi.is\_illegal() \textbf{ or} \textit{ All passengers arrived}$} 
                \State $\textbf{break}$
            \EndIf
        \EndWhile

        \If {$F(\Phi) < F(\Phi_{best})$} 
            \State $\Phi_{best} = \Phi$
        \Else
            \State $\Phi = \Phi_{best}$
        \EndIf  
        \State $t = random\_number\_between(0, t)$
        \State $\varphi = \Phi_{best}.at(t).clone()$
        \State $\Phi.drain(t..)$
    \EndWhile
    \Return {$\Phi_{best}$}
\end{algorithmic}
\end{algorithm}

\begin{algorithm}
\caption{neighbour}\label{alg:neighbour}
\begin{algorithmic}
    \For {$r \in \Theta.train\_ids$}
        \State $\textit{Initalize best move } m_{best}$
        \State $\textit{Get list of legal moves } M_r = legal\_moves(\varphi, r)$
        \For {$m \in M_r$}
            \If {$m > m_{best} \textbf{ and } m \notin T$}
                \State $m_{best} = m$
            \EndIf
        \EndFor
        \If {$m_{best}$}
            \State $\varphi.push(m_{best})$
            \State $T.push(m_{best})$
        \EndIf
    \EndFor
    \Return {$\varphi$}
\end{algorithmic}
\end{algorithm}

\subsection{Evaluating Moves}

An important process of the algorithm is the evaluation of moves. When moves that 
lead to a bad result are excluded early on, a desirable result can be found much 
faster. 

A common evaluation practise is, to choose an objective function - mostly reffered 
to as "cost" or "fitness" function - which for example, maps any tuple of legal 
moves $m$ and possible states $\varphi$ to $x \in \mathbb{R}$:

$$
c : (\varphi, m) \rightarrow \mathbb{R}
$$
 
Another abstraction is to use a set of rules $R$ to define the order of moves. 
We ended up choosing this method as it made surveying the evaluation much easier. 
Pseudocode \ref{is-greater} shows how the comparison is defined:

\begin{algorithm}
\caption{is\_greater}\label{is-greater}
\begin{algorithmic}
    \For {$r \in R$}
        \State $\textit{Get result } \psi = r(a, b, \varphi)$
        \If {$r \neq None$}
            \Return $\psi$
        \EndIf
    \EndFor
    \Return {$false$}
\end{algorithmic}
\end{algorithm}

A rule $r$ is a function $r(a, b, \varphi)$ given by

\[
    r(a, b, \varphi) = 
    \begin{cases}
        true,& \text{if } a \text{ should be prefered to } b\\
        false,& \text{if } b \text{ should be prefered to } a\\
        None,& \text{otherwise}
    \end{cases}
\]

\noindent where $a$ and $b$ are the compared moves.

To simplify the abstraction, rules are divided into categories, each of which 
fills a single purpose. The following sets in the given order are optimized for 
finding an optimal solution for the $large$ data set given by the GI.

\subsubsection{Avoid Station Overload}
\label{rule:1}

The purpose of the rules in this set is to prevent too many trains from arriving 
at stations at the same time, thus overloading the station. 

The first rule compares a departure $a^{depart}$ with the null move $b^0$ and is 
given by

\[
    r(a^{depart}, b^0, \varphi) = 
    \begin{cases}
        false,& \text{if } (c_{est} - 1) \leq -c_{max} \\
        None,& \text{otherwise}
    \end{cases}
\]

\noindent where $c_{est}$ is the estimated station capacity of the arrival station of 
$a^{depart}$, i.e. the capacity of destination at the arrival time of 
the departing train and $c_{max}$ is the maximum capacity of the destination. 
This avoids trains from departing towards overloaded stations.

The second rule compares a train start $a^{start}$ and the null move $b^0$ and is 
given by 

\[
    r(a^{start}, b^0, \varphi) = 
    \begin{cases}
        false,& \text{if } c_{est} \leq 0 \\
        None,& \text{otherwise}
    \end{cases}
\]

\noindent This avoids station overloading caused by starting a train at a station 
to which other trains are en route.

\subsubsection{Detrain Arrived Passengers}
\label{rule:2}

The purpose of the single rule in this set is to detrain passengers from trains that 
have arrived in the destination station of the passenger.

The rule compares a detrain move $a^{detrain}$ to any of 
$b \in \{m^{board}, m^{depart}, m^0\}$ and is given by

\[
    r(a^{detrain}, b, \varphi) = 
    \begin{cases}
        true,& \text{if } s = dest_p\\
        None,& \text{otherwise}
    \end{cases}
\]

where $s$ is the station to which the passenger $p$ should be detrained and 
$dest_p$ is the destination station of $p$.


\subsubsection{Board By Arrival}
\label{rule:3}

The purpose of the single rule in this set is to preferentially board passengers 
with a short arrival time. 

The rule compares boarding $a^{board}$ to boarding $b^{board}$ and 
is given by

\[
    r(a^{board}, b^{start}, \varphi) = 
    \begin{cases}
        true,& \text{if } arrival_a < arrival_b\\
        false,& \text{otherwise}
    \end{cases}
\]

\subsubsection{Board By Travel Path}
\label{rule:4}

The purpose of the single rule in this set is to board passengers together with 
other passengers that travel the same path.

The rule compares boarding $a^{board}$ to any of $b \in \{m^{depart}, m^0\}$ and 
is given by

\[
    r(a^{board}, b, \varphi) = 
    \begin{cases}
        true,& \text{if } \exists (t,s,q) \in \{(t,s,q)_a\} \exists p \in P_t (P(p) \subseteq P(q) \lor P(q) \subseteq P(p))\\
        None,& \text{otherwise}
    \end{cases}
\]

where $t$ is the train, $s$ the destination station and $q$ the passenger to be 
baorded, $P_t$ is the set of passengers travelling in $t$ and $P(p)$ is 
the shortest path from the current location to the destination of $p$.

\subsubsection{Board To Empty Train}
\label{rule:5}

The purpose of the single rule in this set is to board passengers on trains that 
are empty.

The rule compares boarding $a^{board}$ to any of $b \in \{m^{depart}, m^0\}$ and 
is given by 

\[
    r(a^{board}, b, \varphi) = 
    \begin{cases}
        true,& \text{if } c^t_{\varphi} = c^t_{max}\\
        None,& \text{otherwise}
    \end{cases}
\]

where $c^t$ is the capacity of the train $t$ at state $\varphi$, that should be 
boarded to and $c^t_{max}$ is the maximum capacity of the train.

\subsubsection{Depart To Exact Destination}
\label{rule:6}

The purpose of the single rule in this set is to prefer departures that bring the 
train to the destination station of any of its boarded passengers.

The rule compares one departure $a^{depart}$ to another $b^{depart}$ and is 
given by 

\[
    r(a^{depart}, b^{depart}, \varphi) = 
    \begin{cases}
        true,& \text{if } \forall (t,f,d,l) \in \{a,b\} \exists p \in P_t (D(d, d_p) = 0)\\
        None,& \text{otherwise}
    \end{cases}
\]

\noindent where $t$ is the train and $s$ the destination station of the corresponding 
departure. I.e. For all moves $m \in \{a,b\}$, there exists one passenger $p$ in 
train $t$ so that the distance between destination $d$ of the departure and 
destination $d_p$ of passenger $p$ equals $0$.

\subsubsection{Depart Towards Destination}
\label{rule:7}

The purpose of the single rule in this set is to prefer departures that bring the 
boarded passengers of a train closer towards their corresponding destination.

The rule compares one departure $a^{depart}$ to another $b^{depart}$, and is 
given by

\[
    r(a^{depart}, b^{depart}, \varphi) = 
    \begin{cases}
        true,& \text{if } D(d_a, p_0) < D(d_b, p_0)\\
        false,& \text{otherwise}
    \end{cases}
\]

\noindent where $d_m$ is the destination of the corresponding move $m$ and $p_0$ is the 
first passenger in the departed train.

\subsubsection{Depart Passenger Train}
\label{rule:8}

The purpose of the single rule in this set is to depart non empty trains.

The rule compares a departure $a^{depart}$ to the null move $b^{0}$ and is given 
by

\[
    r(a^{depart}, b^{0}, \varphi) = 
    \begin{cases}
        true,& \text{if } \exists (t,f,d,l) \in \{(t,f,d,l)_a,()_b\} (c^t_{\varphi} < c^t_{max})\\
        false,& \text{otherwise}
    \end{cases}
\]

\noindent where $c^t_{\varphi}$ is the capacity of train $t$ at state $\varphi$ and $c^t_{max}$ is the maximum 
capactiy of $t$.

\subsubsection{Depart To Pickup Passenger}
\label{rule:9}

The purpose of the single rule in this set is to depart trains towards 
passengers that have not arrived at their destination station so far.

The rule compares a departure $a^{depart}$ with the null move $b^0$ and is given by

\[
    r(a^{depart}, b^{0}, \varphi) = 
    \begin{cases}
        false,& \text{if } \exists (t,f,d,l) \in \{(t,f,d,l)_a, ()_b\} (P_t = \varnothing \lor P_f \neq \varnothing)\\
        None,& \text{otherwise}
    \end{cases}
\]

\subsubsection{Choose Train Start}
\label{rule:10}

\noindent The purpose of the rules in this set is to start trains at stations 
with passengers. Passengers with low arrival times are prefered.

The first rule compares a train start $a^{start}$ to the null move $b^0$ and is given by

\[
    r(a^{start}, b^{0}, \varphi) = 
    \begin{cases}
        true,& \text{if } \exists (t,s) \in \{(t,s)_a, ()_b\} \exists p \in P_s (|p| \leq c_t)\\
        None,& \text{otherwise}
    \end{cases}
\]

\noindent where $|p|$ is the size of the passenger group $p$. This leads to trains starting 
at stations with passengers that can be boarded on train $t$.

The second rule compares one train start $a^{start}$ to another $b^{start}$ and 
is given by

\[
    r(a^{start}, b^{start}, \varphi) = 
    \begin{cases}
        true,& \text{if } min(arrival_P_a) < min(arrival_P_b) \\
        false,& \text{otherwise}
    \end{cases}
\]

\section{Implementation}

The theory of an algorithm alone is not sufficient to find solutions in a 
desirable time. The implementation must be carried out with the help of suitable 
data structures that make it possible to quickly change and access information. 
This section describes the programming language chosen, design decisions and the 
methods used to improve performance.

\subsection{Rust}

The implementation was done in the multi-paradigm programming language 
\emph{Rust}\footnote{Rust \url{https://rust-lang.org/}}. It is designed 
for performance and safe concurrency. \emph{Rust} uses borrow checkers to validate 
references and thus gurantee memory safety. Due to the high performance it is well 
suited for optimization problems. The well written documentation, the rich 
ecosystem and the integrated package manager \emph{Cargo}\footnote{Cargo \url{https://doc.rust-lang.org/cargo/guide}} 
make it possible to work productively with it.

\subsection{Architecture}

The structure of the implementation follows the layout 
conventions\footnote{Cargo project layout \url{https://doc.rust-lang.org/cargo/guide/project-layout.html}} 
specified for \emph{Cargo} projects. Except for tests, which are located at the 
point of implementation of an abstraction. This design decision was made because 
separating components that involve the same abstractions can increase the complexity 
of a program. Furthermore, abstractions are located following the single-responsibility 
principle in the project root. Rules and the definitions of their order are located 
in the \emph{rules} submodule.

\subsection{Data Structures}

Preparing data through appropriate data structures enables significant performance 
improvements in finding optimal timetables. The following data structures proved 
to work well for the corresponding use cases.

\begin{itemize}
    \item \emph{vector}: A contiguous growable array with an access and update 
    complexity of $O(1)$. It is suitable for holding and adjusting all data of a 
    state $\varphi$.
    \item \emph{hashset}: A hashmap with no value. The complexity $O(k)$, where 
    $k$ is the capacity of the set, allows the quick addition and removal of data.
    Thus, for a state, there are two vectors of hashesets that make it possible 
    to quickly access and update the passengers in a train $t$ or on a station $s$.
    \item \emph{linked-hashset}: A hashset that preverses the insertion order and 
    allows popping the first inserted element in $O(1)$ time. It is used to remove 
    the first state from the tabu list when the maxmium number of states in the set 
    is reached.
    \item \emph{fxhash} \footnote{fxhash \url{https://docs.rs/fxhash/latest/fxhash/}}: 
    The fxhash has its strength in hashing 8bytes on 64-bit platforms and is 
    significantly faster than the standard sip hash which is desgined to withstand 
    "hash flooding" DoS attacks. fxhash is used as a hasher for all hashmaps and 
    hashsets.
    \item \emph{decimal}: For precise calculation with real numbers. It is used 
    to represent the speed of trains and distances between stations.
\end{itemize}

\subsection{Performance}

The precalculation of data allows quick access later on. This saved time by initially
finding the shortest path and distance between all stations using the djikstra 
algorithm. This information is accessed by the rules \hyperref[rule:4]{4}, 
\hyperref[rule:6]{6} and \hyperref[rule:7]{7}.

Furthermore, time could be saved when hashing a state by reducing the attributes 
used for hashing. The location of all trains $L^r$ and passengers $L^p$, the time 
$t$ and the pushed moves $M$ are sufficient to uniquely identify a state, since 
all other attributes can be derived from it. Breaking down the attributes used 
for hashing a state gave a significant performance improvement on large datasets, 
as it made adding and accessing the tabu list less costly.

In addition, benchmarks showed that cloning a state takes a lot of time, especially 
for large models. Thus, performance could be improved by undoing changes to an 
instance of a state instead of remembering a duplicate of its old version.

\section{Results}

Table \ref{table:results} shows the results of the algorithm and the 
implementation in Rust. Two datasets given by the GI are compared, one small 
named "unused-wildcard-train" and a "large" dataset. 

The tests were performed in a Docker container given 2gb memory and 2 cpus. This 
corresponds to the given competition circumstances.

If the search does not improve for 25.000 iterations or does not find the optimum 
within $3$ seconds it is discontinued and considered a failure. An iteration 
is defined as a complete search of a neighborhood.

\begin{center}
\begin{table}[ht]
\caption{Model Results}
\label{table:results}
\begin{tabular}{ l r r }
    \hline\hline
        & unused-wildcard-train & large \\
    \hline\\
    Model: &  & \\[3pt]
    \hspace{3mm}Stations & 5 & 215 \\ 
    \hspace{3mm}Lines & 8 & 18426 \\ 
    \hspace{3mm}Trains & 10 & 37 \\ 
    \hspace{3mm}Passengers & 41 & 721 \\[3pt] 
    Compared moves: &  & \\[3pt]
    \hspace{3mm}Average & 257.089 & 900.465 \\ 
    \hspace{3mm}Maximum & 1.293.599 & 1.362.345 \\ 
    \hspace{3mm}Minimum & 1.578 & 723.708 \\[3pt] 
    Search time: &  & \\[3pt]
    \hspace{3mm}Average & 665ms & 264ms \\ 
    \hspace{3mm}Maximum & 3.001ms & 450ms \\ 
    \hspace{3mm}Minimum & 2ms & 234ms \\ [3pt]
    Compared moves / millisecond: &  & \\ [3pt]
    \hspace{3mm}Average & 458 & 3.256 \\ 
    \hspace{3mm}Maximum & 780 & 3.427 \\ 
    \hspace{3mm}Minimum & 417 & 2.997 \\ [3pt] 
    Optimum reached: & 95\% & 100 \% \\ [3pt]
    \hline
\end{tabular}
\end{table}
\end{center}

\section{Conclusion}

The evaluation of moves through carefully crafted rules combined with tabu-enhanced
genetic search can be used to precisely generate desired timetables. The results 
show that despite the size of the $large$ set, results are found relatively fast 
compared to the $unused-wildcard-train$ set. On the one hand, this shows that 
desirable solution times can easily be reached for large problem sizes, but on 
the other hand, that the rule set is only optimized for a series of inputs similar 
to the $large$ set. Thus, rules can be found that apply better for a largerer range 
of data sets.

Furthermore, it is noticeable that "transit" stations where passengers transfer 
from one train to another are not necessary in order to find optimal solutions. 
This may be different for other data sets. For that case, other rules for 
transfering passengers can be easily added in order to adapt this behavior for 
the timetable as well.

The tabu-enhanced genetic search algorithm thus has the potential to find desirable 
solutions not only for the informaticup but also for the Deutsche Bahn's 
"Taktfahlplan".

\bibliographystyle{plainnat} 
\bibliography{paper}

\end{document}